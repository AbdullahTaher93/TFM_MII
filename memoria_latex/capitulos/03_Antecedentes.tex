\chapter{Antecedentes}
\label{ch:Capitulo 3}

\begin{quote}
  Este capítulo presenta los antecedentes que preceden a este trabajo, así como las líneas de investigación y los trabajos previos centrados en tareas predictivas en el campo de Food Computing.
\end{quote}


\section{Food Computing}

Como ya se ha introducido en el Capítulo \ref{ch:introduccion}, el desarrollo de las nuevas tecnologías y el incremento del interés de la población acerca de la alimentación saludable ha contribuido al aumento en la cantidad de datos generados en este campo. 
Este aumento ha dado pie a su tratamiento con algoritmos que puedan procesar grandes cantidades de información con el fin resolver problemas de interés para la población. Aquí tienen especial protagonismo aquellos que son resultado de la interacción entre usuarios en redes sociales o comunidades de usuarios con interés específico en la cocina para predecir valores de sobrepeso y diabetes en la población~\cite{Fried2015,Abbar2015}. Por otro lado, las páginas web y comunidades de usuarios cuyo objetivo específico es compartir recetas (como pueden ser AllRecipes\footnote{\url{www.allrecipes.com}} o Yummly\footnote{\url{www.yummly.com}}), son hoy en día el origen de la mayor parte de colecciones de datos alimenticios en Food Computing~\cite{min2019survey}. En estas páginas, encontramos multitud de recetas con contenido tanto textual, multimedia, como en muchos casos, nutricional. Debido al gran impacto que tienen, múltipes estudios parten de estos datos para la implementación de herramientas o sistemas en Food Computing~\cite{10.3389/fict.2018.00014,10.1007/978-3-319-02432-5_19}.

Sin embargo, si nos fijamos en estas últimas fuentes de datos mencionadas, cada vez es mayor la tendencia hacia recetas con una fuerte componente no saludables. Hoy en día, la concienciación de la población ha dado lugar al desarrollo de movimientos de vida sana basados en la alimentación, los cuales llevan a las personas a interesarse cada vez más por estas recetas. Aquí reside el interés en el desarrollo de herramientas informáticas que puedan adaptar recetas a necesidades de usuarios, ya sea facilitando dietas completas (o bien recetas) adaptadas al usuario, así como con la creación de las llamadas pseudo-recetas, las cuales se centran en generar recetas en base a unas especificaciones dadas, ya sea de forma completa (generación de recetas completas utilizando por ejemplo redes de sabores y grafos de ingredientes con recetas) o parcial (modificando ingredientes concretos dentro de recetas)~\cite{chen2019eating}. En este trabajo se profundiza en esta segunda línea. 

El otro gran foco de información reside en las bases de datos de composición de alimentos, las cuales permiten acceder a la información nutricional de ingredientes, recetas, o incluso platos preparados~\cite{spanishStandariz}, siendo la base de los estudios alimenticios desde el punto de vista de ciencias como la química y la tecnología de alimentos. En este punto destacan bases de datos como la proporcionada por USDA (United States Department of Agriculture)~\cite{gebhardt2008usda}, la cual es una base de datos de referencia a nivel mundial en cuanto a composición nutricional se refiere~\cite{raper2004overview}. También tienen relevancia las bases de datos de composición nutricional de países europeos disponibles en EuroFIR (European Food Information Resource Network)~\cite{church2009eurofir}.
Ya sea de manera individual o conjunta, estos dos grandes focos de información forman el punto de partida de las principales vías de investigación en Food Computing: cocinas internacionales (y el análisis de dietas y recetas derivadas de dichas cocinas) y estudio de composición de alimentos (mayormente orientado a la tecnología de alimentos). Por tanto, las tareas abarcadas en este área tienen como origen uno de estos dos tipos de fuentes de información (o ambos).

\vspace{0.3cm}
A pesar de la gran extensión de tareas que abarca Food Computing, estas se pueden clasificar en cinco grandes grupos en función de los objetivos que persiguen: percepción, reconocimiento de imágenes, recomendación, predicción y recuperación de información, detalladas a lo largo de esta sección.

\subsubsection{Percepción}
La percepción humana acerca de los alimentos influye de forma directa en los hábitos alimenticios. Por ello, esta tarea es ampliamente estudiada en Food Computing desde el punto de vista de la neurociencia y las ciencias cognitivas. Aquí toma importancia el \textit{factor sensorial} de los alimentos donde se incluyen el sabor, la textura o incluso características superficiales del alimento (p.ej., el color o el brillo). La impresión y las sensaciones que provocan los alimentos influyen en la opinión humana de los que los consumen, existiendo una relación entre éstas y los hábitos alimenticios de la gente. En los últimos años han empezado a surgir líneas de investigación que enfocan estos problemas con Aprendizaje Automático y Redes Neuronales~\cite{ofli2017saki}.

\subsubsection{Reconocimiento de imágenes}
El reconocimiento o identificación de elementos del mundo culinario es otro de los problemas más abordado en este ámbito. En esta línea, el reconocimiento de los denominados \textit{item food}, se ha vuelto una tarea esencial en Food Computing. En concreto, el procesamiento de imágenes es sin duda el mayor foco de atracción en este contexto, el cual ya ha sido abordado de múltiples formas: etiquetado de recetas, identificación de ingredientes y grupos alimenticios en grandes bancos de imágenes, etc. En los últimos años, ha aumentado exponencialmente la resolución de estos problemas a partir de técnicas predictivas. Las últimas tendencias residen en combinar la información obtenida a partir del procesamiento de estas imágenes con información de otras fuentes, dando lugar a un conjunto de información de mayor completitud que permita alcanzar mayor precisión en los resultados.

\subsubsection{Recomendación}
Los sistemas de recomendación forman una de las áreas más explotadas en el campo de la alimentación estos últimos años. Conllevan el desafío de información compleja y polifacética, y esto es lo que la diferencia de las tareas de recomendación centradas en otra áreas, donde puedan existir distintos estándares o componentes más objetivas y, por tanto, más sencillas de calcular e interpretar. Principalmente, han abarcado dos vías que merece la pena destacar. Por una parte, la recomendación basada en las preferencias del usuario en cuestión, teniendo en cuenta para ello sus gustos, sus rutinas, así como patrones en cuanto alimentación que a simple vista puedan ser más complicados de detectar. Por otra parte, se le suma a estos sistemas de recomendación la inmersión en el mundo de la nutrición y vida sana, dando lugar a sistemas que persiguen proporcionar asesoramiento dietético personalizado para el usuario, el cual conlleva una fuerte componente saludable. Está muy ligada a las tareas de predicción, sobre todo, para la parte relativa a la recomendación en función de las preferencias del usuario, lo que suele llevar asociado tareas predictivas~\cite{min2019survey}. En los últimos años, se ha producido un auge en el desarrollo de sistemas de recomendación centrados en la generación de dietas personalizadas teniendo como requisito que sean saludables~\cite{Trattner2017}. 


\subsubsection{Predicción}
La gran cantidad de datos que se produce en este área ha dado lugar a que se empleen técnicas predictivas para resolver problemas de Food Computing. Multitud de parámetros, sobre todo relativos a los alimentos, se han estudiado desde el punto de vista de la predicción.  A pesar de los grandes avances que se han hecho en los últimos años, la complejidad intrínseca en los datos, así como las relaciones entre ellos, hacen que realmente este tipo de tareas tenga éxito en entornos muy controlados y restringidos, y para una cantidad concreta de elementos. En escenarios reales, la variedad de casuísticas y factores a considerar es tan grande que, junto con la falta de estandarización, hace inevitable tener que acotar el problema, reduciendo así las probabilidades de éxito que tendría su aplicación en otras fuentes externas. Entre muchas otras, se ha hecho especial hincapié en la evaluación de propiedades de alimentos, seguridad alimentaria y aspectos culturales. 

Por otra parte, tal y como se ha mencionado en los apartados de tareas relativas a Recomendación y Reconocimiento, su aplicación no queda reducida a un campo concreto de Food Computing, sino que se ve aplicada en múltiples ámbitos. En la Sección \ref{sec_pred} se profundiza en los problemas abarcados en literatura que incluyan técnicas predictivas para su resolución.

\subsubsection{Recuperación de información}
Las tareas relativas a la recuperación de datos alimenticios pueden no tener aplicaciones directas, pero sí son de vital importancia para que el resto de tareas que se han detallado anteriormente en esta sección puedan desarrollarse de manera apropiada y proporcionen resultados de calidad: la recopilación de un dataset extenso de imágenes de platos de comida puede no tener un gran impacto en sí mismo, pero la existencia de estos facilita el diseño y desarrollo de técnicas que sí precisen de grandes cantidades de datos, por ejemplo, para entrenar o validar los modelos obtenidos~\cite{marin2019recipe1m+}. Es por ello que un motor de recuperación de información culinaria se hace indispensable para poder trabajar con estas grandes colecciones de datos.


\section{Técnicas predictivas en Food Computing}\label{sec_pred}

\subsection{Aproximaciones generales}

Desde el punto de vista de la predicción en Food Computing se ha estudiado en amplitud patrones intrínsecos en loa alimentos con objetivos muy dispares. En concreto, se ha hecho especial hincapié en la evaluación de propiedades, sobre todo relacionadas con componentes bioactivos y características psicoquímicas de los mismos. En \cite{Correa2018} se recopilan algunos de los estudios llevados a cabo en este ámbito, como la evaluación de las características antioxidantes de los aceites o el determinar la tasa de fermentación de las semillas de cacao en función de medidas de aminoácidos y de cambios de color en los alimentos. Además de estudiar sus propiedades, el análisis y evaluación de los alimentos en el mercado también han tenido un gran protagonismo en las tareas predictivas realizadas en este campo. Se han llevado a cabo estudios  realizados en base a monitorización con tratamientos térmicos, como medio para determinar estados de buena calidad en productos alimenticios. También se pueden consultar en la literatura de Food Computing algunos estudios acerca de la conductividad térmica de productos de panadería, así como de patrones en base al nivel de deshidratación de frutas, estableciendo una relación con la pérdida de agua. Por otra parte, se han realizado estudios que buscan predecir la relación entre la carga de bacterias y la concentración de las mismas en determinados vegetales, concretamente en el tomate y en las hojas de lechuga \cite{Correa2018}. 

Otro campo explotado desde estas técnicas ha sido el de la seguridad alimentaria, donde se han llevado a cabo estudios que buscan determinar el tiempo de caducidad de determinados productos, así como predecir el estado ideal de refrigeración de platos cocinados, o la calidad de los alimentos en función de su conservación en frío. Los problemas de predicción también han cobrado protagonismo en lo que concierne a tareas de clasificación. En \cite{Correa2018} se propone una metodología para clasificar distintos tipos de aceite, vinagre, o incluso de variedades de queso. También se han estudiado clasificadores de grupos alimenticios a partir de parámetros como pueden ser la claridad, el color, el grado de fermentación o incluso la acidez\cite{min2019survey}. 

Por otra parte, las técnicas basadas en el uso de Redes Neuronales también tienen presencia en el mundo culinario, abarcando principalmente tres vías, que se podrían resumir en predicción de parámetros, clasificación y estudios de calidad. Asimismo, las Series Temporales también tienen cabida en este sector, aunque principalmente han tenido presencia desde un punto de vista médico. En \cite{bellaci-biomedical} se hace uso de estas técnicas para interpretar datos procedentes de la monitorización de pacientes con diabetes, intentando evaluar la salida obtenida con una terapia concreta. Otros estudios exitosos se centran en el estudio de los precios u otras características económicas dentro del mundo de la comida \cite{Zou20072913}. En esta área, también entran estudios relativos a la detección de enfermedades en plantas (en vista a la detección de brotes). En \cite{towards-food-security-ia} estudian la detección de la enfermedad \textit{Blast} en la hoja del arroz a partir de técnicas predictivas en procesamiento de imágenes.

Si nos centramos en el análisis a nivel de receta y no de alimento, es posible aprovechar la detección de ingredientes y los métodos de predicción de la cocina para comparar los alimentos en función de sus componentes~\cite{Singh2015CSE2A}. De manera similar, se puede identificar el país de origen de la receta utilizando términos extraídos del texto~\cite{Min2018YouAreWhat}. Asimismo, otro campo de estudio en el que se han centrado diferentes problemas de predicción ha sido en los aspectos multiculturales relativos al mundo culinario, analizando datos de distintas zonas geográficas a nivel mundial~\cite{Sajadmanesh2019}. La relación entre propiedades de ingredientes, influencia de la región y hábitos alimenticios han sido objeto de estudio en Food Computing durante los últimos años~\cite{Min2018} en lo referente a enfermedades relacionadas con la alimentación. También se han analizado patrones de combinación de ingredientes en distintas regiones~\cite{Bossard2014} para la búsqueda de equivalencias de cocina regional de unas zonas geográficas a otras totalmente dispares~\cite{10.3389/fict.2018.00014}. Sin embargo, los trabajos en este área se han visto limitados por la ausencia de herramientas que permitan gestionar de forma conjunta bases de datos de composición nutricional de distinta procedencia geográfica. Este problema ha sido expuesto en múltiples trabajos~\cite{spanishStandariz}, dando lugar a una necesidad de herramientas de fusión de información de distintas fuentes de datos. La falta de estandarización en los datos y la ausencia de una base de datos estandarizada impide realizar estudios más allá de entornos específicos muy controlados.

\subsection{Modelos predictivos del lenguaje}

Si nos centramos en el uso de técnicas predictivas en Food Computing para el procesamiento de información textual, no han tenido tanto recorrido en comparación con las tareas predictivas mencionadas en la sección anterior. Uno de los trabajos más destacados es \textit{Food2Vec}, donde se utiliza un modelo de Word Embedding entrenado con las listas de ingredientes incluidos en las recetas~\cite{food2vec}. Otro modelo de Word Embedding también entrenado en el ámbito culinario es Recipe2Vec~\cite{recipe2vec}, que aunque codifica todo el texto, se centra en la comparación y recuperación de recetas (además, de no ser código abierto). Sí que se han expuesto las ventajas de un modelo de fusión de información heterogénea en Food Computing, en el que se integren imágenes, textos e información nutricional procedente de recetas~\cite{8099810}. Sin embargo, este último modelo está enfocado a tareas de reconocimiento de imágenes, y los datos textuales que contienen son minoría y no bastan por sí solos para desarrollar un modelo de Procesamiento de Lenguaje Natural.

Además, debemos tener en cuenta que los textos, sobre todo procedentes de recetas suelen incluir marcas de alimentos, ya que, una de las propiedades más características del lenguaje alimenticio es la utilización de manera indistinta de marcas y los nombres de alimentos representados por dichas marcas. En este tipo de textos, a menudo encontraremos marcas sustituyendo a los propios ingredientes. Además, la información de estos productos comerciales también aparece en las bases de datos, por la propia flexibilidad que tienen a la hora de almacenar su información. En consecuencia, nuestro modelo de lenguaje debe permitir lidiar con este tipo de términos. En esta línea, destacamos los trabajos de~\cite{10.1093/jamiaopen/ooz007}, en los que, con un modelo de Word Embedding identificaron, entre otros, nombres de marcas pertenecientes a suplementos dietéticos. Asimismo, se han utilizado modelos de Word Embedding para abordar problemas de evaluación de la calidad de menús de restaurantes, utilizando funciones de puntuación sobre las representaciones vectoriales de platos en los menús de dichos establecimientos~\cite{chao2016dish}.

Más allá de la aplicación de modelos de Word Embedding a recetas y cocinas del mundo, el uso de estos modelos también se ha empleado para detectar similitud a nivel de alimento. Como trabajo destacado en este contexto, se han utilizado representaciones obtenidas con modelos de Word Embedding para enriquecer la red de sabores entre los distintos ingredientes involucrados en recetas, y así poder crear recetas sustitutas en función de factores sensoriales \cite{Sauer2017CookingUF}. En este trabajo, se parte de la hipótesis de que los ingredientes se pueden representar en un espacio de sabores~\cite{ahn2011flavor}, y que a partir de esas representaciones se puede pasar de unos grafos receta-ingredientes a otros, teniendo en cuenta dicha información cognitiva.

Desde una perspectiva más amplia, otros trabajos de investigación estudiaron la relación entre los ingredientes y los métodos de cocción a partir de las descripciones de los datos de los alimentos y redes de sabores. Este último problema ha sido ampliamente abordado en la literatura, sobre todo mediante la aplicación de técnicas personalizadas de procesamiento estadístico de lenguaje natural \cite{Takahashi2012,Chen2017,chang2018recipescape}.

\subsection{Identificación de términos alimenticios textuales}

El problema de mapear términos entre dos fuentes de datos dadas (en nuestro caso, alimenticias) se puede ver como un problema de identificación de términos equivalentes entre dichas fuentes de datos. En las secciones anteriores ya se ha introducido la necesidad de fusionar fuentes de datos de composición nutricional de distinta procedencia. Esta dificultad también ha sido ratificada por otros estudios llevados a cabo sobre estas fuentes de información, los cuales confirman la necesidad de utilizar herramientas informáticas que permitan automatizar la unificación de bases de datos de composición nutricional de alimentos~\cite{barabasi2019unmapped}.

Estas bases de datos se caracterizan por una alta complejidad, nivel de detalle, y periodicidad con la que nueva información de este campo se actualiza (o directamente se genera información nueva), formando el llamado \textit{agujero negro} de la nutrición~\cite{barabasi2019unmapped}. Por ello, el esfuerzo en fusionar fuentes de datos de esta naturaleza se ha centrado fundamentalmente en el mapeo entre atributos de distintas bases de datos nutricionales, con el objetivo de unificar las características nutricionales de los alimentos~\cite{ispirovaG} y así llevar a cabo comparativas a nivel nutricional entre ellas, en este caso, por medio de una ontología.

El uso de ontologías para poder trabajar de forma simultánea con más de una base de datos de composición nutricional ha tenido un largo recorrido~\cite{snae2008foods,dooley2018foodon,6511683}. Un ejemplo de ello es EuroFIR\footnote{\url{http://www.eurofir.org/}} (European Food Information Resource Network), cuya ontología permite, a través de su buscador \textit{FoodExplorer}, mostrar en un formato armonizado el contenido e información nutricional de múltiples alimentos procedentes de bases de datos europeas~\cite{sheehan2008european}. En este ámbito tienen especial protagonismo las ontologías \textit{LanguaL}~\cite{ireland2010langual} y \textit{FoodEx2}~\cite{european2015food}, por su completitud así como por el reconocimiento de los organismos oficiales que las mantienen: US Food and Drug Administration (FDA)\footnote{\url{https://www.fda.gov/}} en el caso de \textit{LanguaL} y European Food Safety Authority (EFSA)\footnote{\url{https://www.efsa.europa.eu/}} en el caso de \textit{FoodEx2}. Sin embargo, la falta de estandarización y los niveles de detalle superficiales alcanzados por dichas ontologías (representan agrupaciones por grupos y subgrupos alimenticios), impiden poder ser utilizadas  para la unificación e identificación a nivel de alimento. Esto hace que las herramientas disponibles de mapeo a nivel de ontología resulten insuficientes para mapear alimentos muy concretos en estas bases de datos~\cite{li2019multi}. 

Desde un punto de vista de tratamiento de información textual, en Food Computing destacan dos vías principales para abordar esta dificultad: el uso de expresiones regulares para extraer información de los alimentos~\cite{chao2016dish}, y utilizar métricas para calcular la distancia entre palabras o textos cortos a partir de modelos de Word Embedding~\cite{Farouk2019MeasuringSS, Kenter2015}, dado el buen funcionamiento que ha tenido a la hora de detectar términos en fuentes de datos externas~\cite{10.1093/jamiaopen/ooz007}. En este contexto, las técnicas de similitud que involucran Lógica Difusa pueden aplicarse para obtener los mapeos entre dos fuentes de datos dadas~\cite{wang2011fast}. Estas técnicas ya han sido probadas anteriormente en el campo de Food Computing para mapear datos alimenticios procedentes de cuestionarios de frecuencia de consumo de alimentos~\cite{10.3389/fnut.2018.00082}.

Más allá de medidas de similitud tradicionales como pueden ser la medida de \textit{Jaccard} o la similitud \textit{Coseno}~\cite{sternitzke2009similarity}, a la hora de detectar equivalencias entre descripciones textuales cortas y modelos de Word Embedding, es destacable mencionar la medida de distancia \textit{Word's Mover}, la cual ha demostrado ser especialmente adecuada para identificar equivalencias a partir de los vectores de documentos cortos obtenidos con Word2Vec~\cite{wmd1,wmd2}.
