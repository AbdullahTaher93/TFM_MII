\chapter{Conclusiones}
\label{ch:Conclusiones}

\begin{quote}
  Este capítulo recoge los principales aspectos concluidos del desarrollo del documento y las futuras líneas de trabajo con las que continuar este proyecto.
\end{quote}


\section{Conclusiones}


% Con la app desarrollada no sólo se puede ver el alcance de la herramienta, sino que se ha podido aplicar a un problema de relevancia que permite ver la gran utilidad que tiene este tipo de herramientas. ( Apartado 4.2.3 )

%  Esto se debe a que esta herramienta se implementa de forma independiente al problema que se intenta resolver con dichos datos, dando lugar a su aplicación en múltiples campos. ( Apartado 4.2.3 ) generalizable a otros campos de estudio. 

Con las representaciones obtenidas con el modelo de Word Embedding entrenado en este trabajo hemos podido ver que utilizar técnicas predictivas de este tipo para obtener representaciones textuales nos ha permitido capturar la semántica de la terminología alimenticia, dando lugar a aproximaciones más precisas que nos permitan tratar con problemas comunes en el lenguaje culinario, como el uso de marcas y sinónimos alimenticios. Al contrario que en otros modelos de Word Embedding en Food Computing, en nuestro caso no sólo utilizamos los ingredientes de las recetas, también usamos el texto que describe las instrucciones de cocción para el entrenamiento del modelo. Con ello, conseguimos obtener codificaciones cercanas para los ingredientes que aparecen juntos en las recetas, pero también para aquellos que están involucrados en preparaciones similares. Esto es útil para la comparación de elementos a nivel multicultural, y también para detectar posibles alternativas o substitutos de alimentos en base a las instrucciones de cocinado empleadas en las recetas. Sin embargo, es importante tener en cuenta que al estar utilizando recetas en el corpus de entrenamiento, corremos el riesgo de obtener muy buenas representaciones de alimentos con mucho protagonismo en el mundo culinario pero pobres o incluso inexistentes para aquellos elementos que no suelan aparecer en las recetas de este tipo de páginas web. Además, es importante valorar que, en este tipo de bases de datos nutricionales, se tienen en cuenta especies florales que no suelen ser consumidas por la población: con nuestro conjunto de entrenamiento, no seríamos capaz de obtener representaciones para este tipo de elementos. Por ello, un conjunto de entrenamiento más amplio, capaz de abarcar alimentos más allá de las recetas, podría subsanar este posible sesgo introducido en el modelo. 

Como se ha podido comprobar con las experimentaciones realizadas, las consideraciones tenidas en cuenta al definir medidas que permitan detectar equivalencias cambian totalmente el comportamiento del módulo de mapeo. Por ello, se han valorado distintos enfoques abarcando tanto información sintáctica y semántica como la vaguedad en el lenguaje que pueda existir en las fuentes de datos utilizadas. Aun así, la similitud entre textos cortos sigue siendo un desafío debido a la dificultad añadida de tratar con su breve contenido, donde se incrementa mucho la importancia que tiene cada palabra perteneciente a la descripción. A pesar de que la información sintáctica y semántica tienen relevancia en este punto (y deben ser tenidas en cuenta en los mapeos) el uso de una métrica que permita darle más importancia a la descripción en sí que a los elementos concretos que forman parte de la misma ha sido determinante para ser capaces de detectar equivalencias entre las bases de datos. No obstante, en los casos de mapeos incorrectos, los resultados han seguido mostrando la calidad y robustez de la herramienta, permitiendo detectar buenas aproximaciones e ingredientes principales. Por otra parte, al analizar los mejores posibles mapeos para cada elemento de i-Diet, hemos podido darnos cuenta cómo no sólo el mejor mapeo posible es de calidad, sino que las 10 primeras alternativas siguen mostrando una mejora muy sustancial en los resultados. Estos mejores mapeos obtenidos con nuestra estrategia, podrían ser combinados con otro tipo de técnicas, o incluso enriquecidos con mayor información del conjunto de datos con el objetivo de alcanzar resultados más exactos.

Con el prototipo implementado para la aplicación de recetas adaptadas hemos podido ver la potencia que podría tener esta aplicación más allá de una solución a un problema de mapeo. Se han podido ver las ventajas de utilizar el modelo predictivo implementado para poder adaptar dietas, puesto que la semántica capturada en el modelo permite alcanzar buenas alternativas a los elementos restringidos. Por otra parte, se ha permitido ver la interpretabilidad de este módulo, ya que paso a paso se indican aquellos ingredientes no adecuados y que deben ser sustituidos por otros. Esta interpretabilidad nos ha permitido percartarnos de la dificultad de algunas adaptaciones, en las que las restricciones limitan las posiblidades de mapeo a alimentos que son difícilmente combinables con los ingredientes de la receta. Esto no deja de ser un problema fuera del ámbito de la computación, ya que la versión vegana o vegetariana de un plato no tiene por qué ser una solución intuitiva de por sí.

En los capítulos correspondientes a cada uno de los módulos implementados, así como en la experimentación llevada a cabo con ellos, se ha podido apreciar que se han tenido en cuenta con éxito los requisitos definidos en el Capítulo \ref{ch:Capitulo 2}. Sin embargo, dada la etapa de desarrollo en la que se encuentra la aplicación móvil (como comentamos en el Capítulo \ref{ch:Consultas_Adaptadas} se trata de un prototipo funcional), el nivel de cumplimiento de los requerimientos relativos a la interfaz podría mejorarse con la realimentación que se obtiene de las pruebas y evaluación con usuarios en etapas posteriores de desarrollo.

Por último, puntualizar que las distintas tareas de Food Computing involucradas en el problema que hemos abordado en este trabajo (Predicción, Recomendación, Identificación y Recuperación de información) son tareas generales que se pueden encontrar en otros campos de estudio que no tienen por qué estar centrados en el ámbito culinario y de la nutrición. Al tratarse de tareas comúnmente abordadas en cualquier campo, justifican aún más la independencia del sistema desarrollado al área de Food Computing, posibilitando su aplicación en otros muchos contextos. Con ello, queremos destacar la relevancia de esta aproximación para resolver problemas que lidien con información heterogénea, puesto que todo el procedimiento puede ser aplicable a otro campo de estudio, cambiando las fuentes de datos y  el modelo de Word Embedding por uno específico en el área en cuestión o incluso por uno genérico si la situación lo propiciase.


\section{Trabajo futuro}

A lo largo del trabajo, hemos podido ver las grandes posibilidades que brindan los modelos predictivos a la hora de trabajar con modelos de lenguaje. En este punto se abren nuevas vías que pretendemos estudiar para poder obtener un modelo del lenguaje más sofisticado. 
Como es de esperar en un problema de predicción, una vía de mejora reside en la calidad de los datos utilizados. Las dificultades que nos hemos encontrado a causa de traducciones inexactas e incorrectas en los datos han derivado en representaciones del lenguaje imprecisas. Una mejora de las traducciones de los alimentos, así como de errores tipográficos, permitiría reducir el ruido añadido, y obtener mejores resultados: cuanta mayor calidad tengan los datos textuales con los que trabajemos, con mayor facilidad encontraremos buenas representaciones con el modelo. Por ello, también planteamos el uso de técnicas de Traducción Automática (Machine Translation) para mejorar la calidad de estos y así reducir el ruido añadido en los datos. Además, aumentar la cantidad de datos alimenticios abarcados en el conjunto de entrenamiento para el Word Embedding nos permitirá obtener un vocabulario más amplio. Para ello, estudiaremos obtener un corpus de entrenamiento más extenso más allá de la información que nos puedan proporcionar las recetas, para así poder recoger tanto vocabulario alimenticio como nos sea posible. Además de aumentar el tamaño del corpus de entrenamiento, como trabajo futuro se plantea utilizar otros modelos de lenguaje más actuales como BERT~\cite{devlin2018bert}, así como aplicar Trasferencia de Aprendizaje~\cite{pan2009survey} (Transfer Learning) con el modelo ya implementado con el objetivo de obtener representaciones aún más precisas que nos permitan convertir los mapeos aproximados (aunque erróneos) en correctos.

Por otra parte, trabajar con modelos de Word Embedding en Food Computing nos ha llevado a comprobar la importancia de la influencia cultural existente al utilizar las distintas cocinas del mundo. Esta dificultad abre nuevas vías de estudio con modelos de lenguaje en Food Computing, tales como abordar problemas de modelos de Word Embedding multilingües, más conocidos como \textit{Cross-Lingual Word Embedding}~\cite{ruder2019survey} en vistas a poder lidiar de forma más sofisticada con las características culturales contenidas en los datos con los que hemos trabajado en este proyecto. Con ello, podríamos detectar posibles transformaciones lineales de los datos alimenticios entre zonas geográficas y abordar de forma más sofisticada las dificultades intrínsecas al trabajar con bases de datos nutricionales procedentes de distintas culturas.

A su vez, confiamos en que transformar este problema en uno multimodal que tenga en cuenta el resto de información contenida acerca de cada alimento (como puede ser el grupo de alimento o incluso los macronutrientes) nos permita obtener un procedimiento de mapeo más robusto, así como una mayor precisión en los resultados. La información que acompaña a los alimentos nos puede ayudar a mejorar la calidad de los mapeos, ya que estaríamos teniendo en cuenta descripciones más completas de los elementos a mapear. Para ello, pretendemos estudiar el problema desde dos perspectivas: en primer lugar, considerar un modelo predictivo que combine la representación vectorial de las descripciones con el resto de datos asociados a cada elemento; y en segundo lugar, considerar estos datos adicionales a la descripción textual en la función de distancia utilizada para medir la similitud entre los alimentos. 
 
